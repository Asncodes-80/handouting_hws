\documentclass{article}

\usepackage[utf8]{inputenc} % allow utf-8 input
\usepackage[T1]{fontenc}    % use 8-bit T1 fonts
\usepackage{hyperref}       % hyperlinks
\usepackage{url}            % simple URL typesetting
\usepackage{booktabs}       % professional-quality tables
\usepackage{amsfonts}       % blackboard math symbols
\usepackage{nicefrac}       % compact symbols for 1/2, etc.
\usepackage{microtype}      % microtypography
\usepackage{graphicx}
\usepackage{xepersian}
\settextfont{XB Roya}

\renewcommand{\baselinestretch}{1.25}

\title{آزمایشگاه پایگاه داده}

\author{%
  علیرضا سلطانی نشان\\
  دانشکده شمسی پور\\
  مهندسی نرم افزار\\
  \texttt{asn80.asn@hotmail.com}
}
\begin{document}
\maketitle
\tableofcontents

\newpage

\begin{minipage}{0.1\textwidth}% adapt widths of minipages to your needs
  \includegraphics[width=1.1cm]{tvu.png}
  \end{minipage}%
  \hfill%
  \begin{minipage}{0.9\textwidth}\raggedleft
  دانشکده شمسی پور\\
  گزارش پروژه نهایی درس پایگاه داده\\
  \end{minipage}

\par\noindent\rule{\textwidth}{3pt}

{\par\centering\large گزارش \par}

{\par\centering\large انجام پروژه پایگاه داده دانشگاهی\par}

\par\noindent\rule{\textwidth}{1pt}


\begin{abstract}
پایگاه داده دانشگاهی در حد مطلوب پیاده سازی شده است و تا حدود مورد بررسی دقیق قرار گرفته است که 
تقریبا قابلیت پیاده سازی حقیقی را داشته باشد.
\end{abstract}

\section{مقدمه}
لازم به ذکر است که تمام فعالیت های انجام شده در این پروژه در فضای
\lr{MySQL Server}
بوده است که هم شِما و هم دیتا با آن مورد بررسی قرار گرفته و سپس دیاگرام کلاس آن
به صورت کاملا خودکار توسط نرم افزار رسم شده.

\section{کار‌های مرتبط / پیش‌زمینه}
باتوجه به آن که این باری دوم است که درسی تحت این عنوان را اخذ کردیم
این اولین پروژه ای است که در قالب محیط عملیاتی دانشگاه 
توسط بنده پیاده سازی شده، که بنده مهارت های تشخیص چگونگی مسئله،
چگونگی پیاده سازی، درک مسئله هم در قسمت پایگاه داده و هم در قسمت بک اند را، بر خوردار هستم.



{\par\large \textbf{ حالا چه کار هایی صورت گرفته است؟} \par}

به طور کلی تا این بخش 5 جدول اصلی وجود دارد:\\
\subsection{جدول دانشجویان}
جدولی که موجودیت های اصلی در آن حضور دارند، که هر دانشجو حاوی یک
شماره دانشجویی میباشد و همچنین دارای یک کدملی منحصر به فرد، علاوه بر
نام و نام خانوادگی و نام پدر، مشخصات دیگری در این جدول ایجاد شده مانند آنکه
این دانشجو چند ساله است که نوع داده ای آن از نوع عدد میباشد که بتوانیم در نهایت
یکسری اطلاعات را بوسیله آن بازیابی کنیم.\\

\begin{minipage}{0.1\textwidth}% adapt widths of minipages to your needs
  \includegraphics[width=1.1cm]{tvu.png}
  \end{minipage}%
  \hfill%
  \begin{minipage}{0.9\textwidth}\raggedleft
  دانشکده شمسی پور\\
  گزارش پروژه نهایی درس پایگاه داده\\
  \end{minipage}

\par\noindent\rule{\textwidth}{3pt}



فیلد بعدی در مورد جنسیت دانشجو است که به صورت 
بولین مورد استفاده قرار گرفته. 0 یعنی دختر بودن و 1 یعنی پسر بودن.\\

زمانی که در دانشگاه پذیرش شده است که از نوع زمانی است، و همچنین زمان فارغ التحصیلی اون که از نوع داده تاریخ زمانی می باشد.
زمان فارغ التحصیلی تا زمانی که 
دانشجو مقطع کنونی را به پایان نرسانده باشد، مقدار ندارد و برابر نال خواد بود.
علاوه بر شهر و آدرس محل زندگی داشنجو، فیلدی برای ذخیره نمره پایانی دیپلم دانشجو تعریف شده است.
\\
ایمیلی که به بخش های مختلف میتواند برای دسترسی به دانشجو، کمک کند.\\

دو بخش اصلی که نباید از آنها گذشت آن است که در چه رشته ای تحصیل میکند، کامپیوتر حسابداری
یا الکترونیک. من این فیلد را از نوع 
enum 
تعریف کرده ام (باتوجه به رشته های موجود در دانشگاه خود)
و مورد بعدی،
فیلد 
\lr{sub study}که مربوط به گرایش درسی دانشجویی است که در آن رشته تحصیل میکند.



\subsection{جدول اساتید}
\lr{prof id}
از اسمش معلوم است که مربوط به شماره منحصر به فرد استاد میباشد که علاوه بر کد ملی اون تعریف شده است.\\

\lr{prof degree}
نکته قابل توجهی دارد، اول اینکه رشته اصلی استاد چیست؟ و دوم اینکه ممکن است استاد مربوطه چندین
مدرک تحصیلی داشته باشد که از آن فارغ التحصیل شده است و میتواند در آن رشته هاهم 
تدریسی داشته باشد، به همین خاطر جدولی دیگر هم برای آن ایجاد کردم تحت عنوان اسم
\lr{prof degrees}


فیلد مهم دیگر،
\lr{prof background years}
است که مشخص میکند استاد مورد نظر چندسال سابقه دارد، در صورتی که بیشتر از عدد مورد نظر باشد نمیتواند به عنوان
استاد اصلی تدریس کند بلکه باید توسط مسئول آن مورد بررسی قرار بگیرد.



\subsection{جدول دروس}
جدول دروس، ویژگی های خاصی ندارد، بطور کلی دارای یک کد درس، یک نام درس و یک واحد میباشد.
اما بخش اصلی آن که تقریبا مهم میباشد
\lr{cour duration}
که مربوط به مدت زمانی است که درس باید تدریس شود.
و 
\lr{sub study}
که از نوع enum
میابشد که این درس برای چه گرایشی مورد تدریس قرار میگیرد.


\begin{minipage}{0.1\textwidth}% adapt widths of minipages to your needs
  \includegraphics[width=1.1cm]{tvu.png}
  \end{minipage}%
  \hfill%
  \begin{minipage}{0.9\textwidth}\raggedleft
  دانشکده شمسی پور\\
  گزارش پروژه نهایی درس پایگاه داده\\
  \end{minipage}

\par\noindent\rule{\textwidth}{3pt}


\subsection{جدول کارمندان}
تا به قبل این بخش درباره موارد اصلی درسی صحبت کردم، الان به بخش اداری آن
می پردازم.\\

کارمندان حاوی یک مشخصه منحصر به فرد میباشند به علاوه کد ملی خود.\\
تحصیلات و میزان آن که به صورت enum
باید انتخاب شود که آیا دیپلم است یا فوق دیپلم، فوق لیسانس، دکتری و غیره.\\
علاوه بر سن و و تولد و محل زندگی، فیلدی تحت عنوان وضعیت تاهل تعریف شده که به صورت بولی صفر و یک از آن استفاده میشود.\\

قسمت های اصلی که به نظرم لازم است توضیح خوب داده شود مربوط به فیلد های 
\lr{cooperation date}
و
\lr{resgination data}
است، در هر دو در مورد زمان ورود و خروج از دانشگاه را مورد بررسی قرار میدهیم،
یعنی فعلا نگفتم که این فرد کدام واحد کار میکند بلکه در صفحات بعدی بیشتر صحبت میکنم.


\subsection{بخش ها و واحدهای مختلف دانشگاه}
خب، یک دانشگاه که یک بخش ندارد، بلکه دارای بخش ها و واحد های مختلفی مانند 
ریاست، آموزش، کارگزینی، امورمالی، دریافت پرداخت، انبار داری وغیره میباشد، به همین خاطر
یک جدول را تعریف کردم که هر بخش به همراه ایدی خود در آنجا مشخص شده.\\

 جداول دیگری که قصد دارم توضیح بدم در رابطه با جداول ارتباط دهنده است که بایستی اشاره کنم که، 
 هر قسمتی از جدولی که حاوی کلید اصلی میباشد در این جداول ارتباطی به صورت کلید خارجی استفاده شده است، تا بتوانیم با استفاده از 
 پیوندها
 \footnote{Join}
 استفاده میکنیم تا بازیابی اطلاعات را به راحتی انجام دهیم.

\subsection{جدول ارتباطی بین دروس و اساتید}
همانطور که خودتان میدانید در سیستم ناد یک قسمتی وجود دارد که ما
میتوانیم از طریق آن متوجه شویم که چه اساتید چه دروسی را ارئه میدهند که توسط
شماره ارائه آنها، آن درس با آن استاد را اخذ کنیم.\\
جدول 
\lr{provide course}

\begin{minipage}{0.1\textwidth}% adapt widths of minipages to your needs
  \includegraphics[width=1.1cm]{tvu.png}
  \end{minipage}%
  \hfill%
  \begin{minipage}{0.9\textwidth}\raggedleft
  دانشکده شمسی پور\\
  گزارش پروژه نهایی درس پایگاه داده\\
  \end{minipage}

\par\noindent\rule{\textwidth}{3pt}

که یک فیلد برای نشان دادن کد ارائه درس میباشد که منحصر به فرد است، و در آن ایدی های اساتید و درس هایی که آنها انتخاب کرده اند به همراه ترمی که میخواهند آنرا ارائه بدهند
استفاده شده است.

\subsection{جدول ارتباطی بین دروس و دانشجویان}
این جدول یک جدولی با اهمیت بالاست، چرا که توسط این جدول و جدول قبلی که توضیح دادم میتوانید به اطلاعات مهمی دست پیدا کنید.
این جدول که نامش
\lr{std optained units}
از اسمش معلوم است که مشخص میکند که هر دانشجویی چه درسی را اخذ کرده است.
حاوی یک سریال از نوع ایدی خودکار است، به همراه شماره دانشجویی و کد درس و ترمی که 
دانشجو آن درس را اخذ کرده است.\\

برای اینکه بتوانیم توسط این جدول و جدول قبلی ببینیم که چه دانشجویی چه درس هایی را با چه اساتیدی اخذ کرده میتوانیم از پیوند بین این چند جدول استفاده کنیم.\\

بازم هم لازم میبینم که بگویم در تمامی جداول اصلی، هر کلید اصلی که وجود  داشت، در جداول ارتباطی به عنوان کلید فرعی یا خارجی با مرجعی به جدول اصلی تعریف شده که از نظر
ارتباطات جداول نگرانی خاصی نداشته باشیم.

\subsection{ارزیابی نهایی}
به طور کلی این جدول یک جدول همه فن حریف است، که به تمامی جداولی که حکم تدریس و تحصیل را داشته اند،
میتواند راه و پیوند بخورد، که از قسمت هایی مانند، 
شماره دانشجویی، 
کد درس، 
کد استاد،
نمره ای که از آن درس کسب کرده است، 
ترم مشخص شده که توسط پیوند به جداول ارتباطی دیگر میتوان بدست اورد.


\subsection{جدول ارتباطی بین کارمندان با بخش های مختلف}
این جدول، جدول اخر و مربوط به بخش اداری دانشگاه میباشد، 
بطور کلی ارتباط بین کارمندان و بخش های مختلفی از دانشگاه است که در بخش های قبلی
به شما توضیحش داده شده.\\

میگوید که کارمندی با یک شماره کارمندی در بخشی با یک شماره بخش مشخص تحت عنوان
(ریئس یا کارمند زیر دست)
در تاریخ مشخص آمده است و در تاریخی دیگیر که میتواند دلایل دیگری داشته باشد، مثل
\\
\hfill \break 
\begin{minipage}{0.1\textwidth}% adapt widths of minipages to your needs
  \includegraphics[width=1.1cm]{tvu.png}
  \end{minipage}%
  \hfill%
  \begin{minipage}{0.9\textwidth}\raggedleft
  دانشکده شمسی پور\\
  گزارش پروژه نهایی درس پایگاه داده\\
  \end{minipage}

\par\noindent\rule{\textwidth}{3pt}\\

فوت، ترفیع درجه و موقعیت از بخشی به بخش دیگر منتقل یا خارج شده باشد، درج شده است.






\section{مدل پشینهاد شده}
در این پروژه برای نمایش برخی از جریان کارها از دیاگرام های کلاسی استفاده کردم، 
همچنین از نرم افزار
\lr{xmind}
و خود
\lr{MYSQL Server}
برای نمایش 
\lr{UML}
\footnote{Unified Model Language}
استفاده میکنم.

\section{نتایج}
نتایج کار را میتوانید در فایل های ضمینه شده یا در صفحه گیت هاب بنده در قسمت پایگاه داده ترم سوم مشاهده کنید، که در بخش مراجع و منابع لینک ها موجود میباشد.



\newpage
\section*{منابع}

\medskip

\small
\LTR 
\latin

[1] \href{https://github.com/Asncodes-80/handouting_hws/tree/main/Database%26az/Term3/prictice/universityProject}{My github}page you can click on, for show all things about, University database project.


\end{document}